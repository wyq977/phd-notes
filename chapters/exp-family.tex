
\section{One-dimensional exponential case}

\begin{align*}
    &\Gc := \{g_{\eta}(y), \eta \in A,  \in \Yc \}, \quad A \text{ and } \Yc \in \Rf^p \\
    &g_{\eta}(y) := \exp\left(\eta y - \psi(\eta)\right) g_0(y) m(\dd{y})
\end{align*}

$y$ is the \textit{sufficient} or \textit{natural} statistics. 
$y = y(x)$ is a function of an observed data $x$.

$m(\dd{y})$ is the \textit{carrying measure} like the discrete measure 
($1$ on non-negative integers for Poisson family) or the uniform measure on $[-\infty,\infty]$
for Gaussian family.

$g_0(y)$ is the \textit{carrying density}.

$\psi(\eta)$ is the \textit{normalizing function} or \textit{cumulant generating function} (CGF).
It scales the $g_{\eta}(y)$ back to $1$ over the sample space $\Yc$.

\begin{remark}[Carrying density in p.3 efron]
    Any $g_{\eta_0}(y) \in \Gc$ could be the carrier density and the members of $\Gc$
    are absolutely continuos with respect to each other, 
    i.e. their null measure sets agree.
\end{remark}

\begin{remark}
    \textit{Cumulant generating function} for $\psi(\eta)$ originates from
    the old techniques for finding expectations, variances and higher-order moments.
\end{remark}

\subsection{Moment Relationships}

We can differentiate $\exp(\psi(\eta)) = \int_{\Yc} \exp(\eta y) g_0(y)m(\dd{y})$

\begin{align}
    \dot{\psi}(\eta) \exp(\psi(\eta)) &= \int_{\Yc} y\exp(\eta y) g_0(y) m(\dd{y}) \\
    \ERWi{g_0}{y} = \dot{\psi}(\eta) &= \int_{\Yc} y\exp(\eta y - \psi(\eta)) g_0(y) m(\dd{y}) 
\end{align}

Differentiating $\exp(\psi(\eta))$ twice gives:
\begin{align}
    (\ddot{\psi}(\eta) + \dot{\psi}(\eta)) \exp(\psi(\eta))
        &= \int_{\Yc} y^2 \exp(\eta y) g_0(y) m(\dd{y}) \\
    \VAR{y} = \ddot{\psi}(\eta) &= \int_{\Yc} (y^2 - y)\exp(\eta y - \psi(\eta)) g_0(y) m(\dd{y})
\end{align}

\section{Exponential family}

\textit{The notation from Hao's thesis is used here for consistency, 
this section is mainly on Peter's MDL and Bradley Efron's book.}

\subsection{Unifying the Notation Hell}

Efron's book (Chap. 2)
\begin{align}
    &\Gc := \{g_{\eta}(y), \eta \in A, y \in \Yc \}, \quad A \text{ and } \Yc \in \Rf^p \\
    &g_{\eta}(y) := \exp\left(\eta^\transpose y - \psi(\eta)\right) g_0(y) m(\dd{y})
\end{align}
\begin{itemize}
    \item $y$: \textit{sufficient statistics}, $p \times 1$ vector in $\Yc \subset \Rf^p$,
    \item $\eta$: \textit{canonical parameter}, $p \times 1 $ vector in $A \subset \Rf^p$
    \item $g_0(y)$: \textit{carrying density} w.r.t. some \textit{carrying measure} $m(\dd{y})$ on $\Yc$
    \item $A$ is the \textit{canonical parameter space}: $\int_{\Yc} \exp(\eta\transpose y) g_0(y)m(\dd{y}) <\infty$,
    \item $\psi(\eta)$ is the \textit{normalizing function} or \textit{cumulant generating function} (CGF)
\end{itemize}

Hao's
\begin{align}
    &p_{\bm{\beta}; \bm{\mu}^*}(u) := \frac{1}{Z_p(\bm{\beta}; \bm{\mu}^*)} 
        \exp\left( \bm{\beta}^\transpose t(u) \right) \cdot p_{\bm{\mu}^*}(u)
\end{align}
\begin{itemize}
    \item $u$: random variable in $\Uc \subset \Rf^d$ with some measure $v$
    \item $t(u)$: \textit{sufficient statistics} for some measurable function $t$
    \item $\bm{\beta}$: \textit{canonical parameter}, $d \times 1 $ vector in $\Bt$
    \item $Z_p(\bm{\beta}; \bm{\mu}^*)$: \textit{partition function}, 
        $\int \exp(\bm{\beta}^\transpose t(u)) p_{\bm{\mu}^*}(u)\dd{v} < \infty$
    \item $p_{\bm{\mu}^*} = p_{\bm{0}, \bm{\mu}^*} \quad q_{\bm{0},\bm{\mu}^*} = q = q_{\bm{\mu}^*}$: why?
\end{itemize}
\Yongqi{It is rather simple when you write down $\bm{\beta}=\bm{0}$}

\begin{align}
    p_{\bm{0};\bm{\mu}^*}(u) &:= 
        \frac{1}{Z_p(\bm{0};\bm{\mu}^*)}\exp(\bm{0}^\transpose t(u))\cdot p_{\bm{\mu}^*}(u) \\
        &= \frac{p_{\bm{\mu}^*}(u)}{Z_p(\bm{0};\bm{\mu}^*)} \\
        &= \frac{p_{\bm{\mu}^*}(u)}{\int\exp(\bm{0}^\transpose t(u)) p_{\bm{\mu}^*}(u)\dd{\nu}} 
        = p_{\bm{\mu}^*}(u)
\end{align}

A naive question follows: what is the distribution $p_{\bm{\mu}^*}(u)$? 
Is it in the mean-value parameterization or canonical form?

Now next to MDL book:


\begin{define}[Exponential Family]
$\PRi{\beta}{X} := \exp(\beta^T\phi(X) - \psi(\beta)) r(X)$
\end{define}
where $r(X)$ is known as the carrier function.

We can $Z(\beta)$ the partition function for all $\beta \in \Theta_{\textsc{can}}$,
defined as 
\begin{align*}
    Z(\beta) := 
\end{align*} 



\begin{itemize}
    \item In MDL, an exponential family is an \textit{open convex} exponential family 
        in a more general sense.
    \item An exponential family is \textit{full} 
        if $\ThetaCan$ contains all $\beta$ s.t. $Z(\beta) < \infty$.
        $\ThetaCan$ is called the \textit{natural parameter space}
\end{itemize}



\textbf{Why Hao explicitily denote the mean of the carrier density in \cite{grunwaldSafeTesting2024}}
