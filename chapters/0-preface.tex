% \printclassoptions
\newpage

% Just something to cite \cite{Mittelbach2004}.

\section{\LaTeX\ Project Management}

The goal is to have a consistent boilerplate for \LaTeX projects,
I choose AOS for regular article.\sidenote{\url{https://vtex-soft.github.io/texsupport.ims-aos/}}

\begin{itemize}
    \item \verb|chapters/01-*.tex|: individual files
    \item \verb|fig|: figures to reproduce
          \begin{itemize}
              \item External figures by \textsf{R} or Python, DPS $\geq 300$, \texttt{.pdf}
              \item TikZ: \texttt{.tex} and \texttt{.pdf}
              \item Asymptote: \texttt{.asy} and \texttt{.pdf}
          \end{itemize}
\end{itemize}

Below is an example project hosted on Github or Overleaf:

\begin{Verbatim}
    ├── chapters
    │   ├── 01-preface.tex
    │   ├── 02-intro.tex
    ├── fig
    │   ├── hilbertrecursive.tex
    │   ├── hilbertrecurves.pdf
    │   ├── helix.asy
    │   ├── helix.pdf
    ├── latexmkrc
    ├── main.bib
    ├── main.pdf
    ├── main.tex
    └── tex
        ├── macro.tex
        ├── tufte-book.cls
        ├── tufte-common.def
        ├── tufte-handout.cls
        └── tufte.bst
\end{Verbatim}

\section{Mathematical Notation}

It has always been a hassle to organise mathematical notation across different sources,
in fact, I would go so far as to argue that it is the most annoying thing
when one starts reading a book or an article.

However, there \textit{must be} some notational conflicts beyond primary school
simply due to the fact that the limited number of alphabets (\textbf{26}).
For example, ``$\mathbb{E}$'' might be energy in physics
while it could refer to expectation or scores in probability.

Another difficulty is that  the authors often assume some familiarity in the topics
\textit{also} I am expected to read in some logical or chronological order.
In reality, I am constantly jumping back and forth between one literature to another.

\begin{table}[h!]
    \centering\renewcommand{\arraystretch}{1.2}
    \begin{tabular}{ccl}
        \textbf{Symbol} & \textbf{Context} & \textbf{Meaning}                       \\
        $\mathbb{Z}$    &                  & set of integers                        \\
        $\mathbb{R}$    &                  & set of real numbers $(-\infty,\infty)$ \\
    \end{tabular}
    \label{tab:real-notation}
\end{table}