% \printclassoptions
\newpage

% Just something to cite \cite{Mittelbach2004}.

\section{\LaTeX\ Project Management}

A consistent boilerplate for \LaTeX projects,
I choose AOS for regular articles.\url{https://vtex-soft.github.io/texsupport.ims-aos/}

The references are managed externally by Zotero and BBT,
exported to BibTex format,
then included via \verb|natbib|.
Below is an example project hosted on Github or Overleaf:

\begin{Verbatim}
    ├── chapters
    │   ├── 01-blabla.tex
    │   ├── ...
    ├── fig
    │   ├── R/Python.pdf
    │   ├── TikZ.tex
    │   ├── TikZ.pdf
    │   ├── Asymptote.asy
    │   ├── Asymptote.pdf
    ├── latexmkrc
    ├── main.bib              % References
    ├── main.pdf              % Main output
    ├── main.tex              % Main document
    └── tex
        ├── macro.tex         % All my collected macros
        ├── custom-style.cls
        ├── custom-style.def
        ├── custom-style.sty
        └── custom-style.bst
\end{Verbatim}

\subsection{Figure sizes and font sizes}

Ratio, margin, font size, how to adjust accordingly...

\subsection{Reference style}

\newpage

\section{Mathematical Notation}

It has always been a hassle to organise mathematical notation across different sources,
in fact, I would go so far as to argue that it is the most annoying thing
when one starts reading a book or an article.

However, there \textit{must be} some notational conflicts beyond primary school
simply due to the fact that the limited number of alphabets (\textbf{26}).
For example, ``$\mathbb{E}$'' might be energy in physics
while it could refer to expectation or scores in probability.

Another difficulty is that  the authors often assume some familiarity in the topics
\textit{also} I am expected to read in some logical or chronological order.
In reality, I am constantly jumping back and forth between one literature to another.

\begin{align*}
    \Af, \Bf, \Cf, \Df, \Ef, \Ff, \Gf, \Hf, \Jf, \Kf, \Lf, \Mf, \Nf, \Of, \Pf, \Qf, \Rf, \Sf, \Tf, \Uf, \Vf, \Wf, \Xf, \Yf, \Zf      \\
    \Ac, \Bc, \Cc, \Dc, \Ec, \Fc, \Gc, \Hc, \Ic, \Jc, \Kc, \Lc, \Mc, \Nc, \Oc, \Pc, \Qc, \Rc, \Sc, \Tc, \Uc, \Vc, \Wc, \Xc, \Yc, \Zc \\
    \Ak, \Bk, \Ck, \Dk, \Ek, \Fk, \Gk, \Hk, \Ik, \Jk, \Kk, \Lk, \Mk, \Nk, \Ok, \Pk, \Qk, \Rk, \Sk, \Tk, \Uk, \Vk, \Wk, \Xk, \Yk, \Zk
\end{align*}
$$
    \arginf, \argsup, \argmax, \argmin, \conv
$$

This stackexchange answer \footnote{\url{https://tex.stackexchange.com/a/58124}}
is probably the most comprehensive answer to which fonts are shown in \LaTeX.

\begin{table}[h!]
    \centering\renewcommand{\arraystretch}{1.2}
    \begin{tabular}{ccl}
        \textbf{Symbol} & \textbf{Usage} & \textbf{Comments}                                 \\
        $\Bf$           & \verb|\*f|     & blackboard bold except \verb|\If| due to conflict \\
        $\Bc$           & \verb|\*c|     & calligraphic font                                 \\
        $\Bk$           & \verb|\*k|     & fraktur font                                      \\
    \end{tabular}
    \label{tab:macro-letters}
    \caption{Macros for Letters}
\end{table}