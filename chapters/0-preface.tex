% \printclassoptions
\newpage
\section{Mathematical Notation}

It has always been a hassle to organise mathematical notation across different sources,
in fact, lots of my peers argue that it is the most annoying things
when one starts reading a book or an article.

However, there \textsc{must be} some notational conflicts beyond primary school
simply due to the fact that the limited number of alphabets.
For example, $\mathbb{E}$ might be energy in physics while it could mean
expectation or scores in the context of probability.

One reasons that I found it difficult is that the authors often assume
some familiarity in the topics \textsc{also} readers should be reading
in some logical or chronological sequences while in reality, I,
am usually jumping between one literature to another.

\begin{table}[h!]
    \centering\renewcommand{\arraystretch}{1.2}
    \begin{tabular}{ccl}
        \textbf{Symbol} & \textbf{for ...} & \textbf{Meaning}                             \\
        $\mathbb{R}$    &                  & set of real numbers $(-\infty,\infty)$       \\
        $\mathbb{R}$    &                  & set of nonnegative real numbers $[0,\infty)$ \\
    \end{tabular}
    \label{tab:real-notation}
\end{table}
\newpage