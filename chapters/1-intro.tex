\clearpage

\section{Reverse Inverse Projections (RIPr)}

\subsection{Li's Algorithm}

Originally developed by @liEstimationMixtureModels1999,
the following algorithm is written based on the notes from \cite{grunwaldSafeTesting2024}
@haoEvalues$k$sampleTests2024. 
To quote Brinda, Li's inequality requires the family $\mathcal{Q}$ to have a
uniformly bounded density ratio.

Li obtains the RIPr in a greedy manner where the KL divergence between
distribution $Q$ onto the convex hull of a set of distributions $\mathcal{Q}$
(composite null) is minimized. It is assumed that the KL divergence between $Q$
and any distribution $Q \in \mathcal{Q}$ is finite 
(often call ``nondegenerate'' condition).

\begin{algorithm}
    \DontPrintSemicolon
    \caption{Li's Algorithm}
    \BlankLine
    $Q_{(1)} = {\arg\min}_{Q \in \mathcal{Q}} D(P \Vert Q)$\;
    \For{$m = 2, 3, \dots, K$}{
        $Q := \alpha Q_{(m-1)} + (1 - \alpha) Q'$\;
        $\alpha, Q' \leftarrow \arg\min_{\alpha, Q'} D(P \Vert Q)$
    }
\end{algorithm}

Here, the distribution $Q'$ and $\alpha$ is chosen (coupled) such that the divergence is minimized. 
The minimizer need not be unique.\sidenote{I think}

\textbf{Regularity condition on $\mathcal{Q}$}

Li's algorithm is apparently greedy with high fluctuation in initial steps. 
Additionally, this task is computationally expensive, and it is not clear of the convexity.

\textbf{Is the returned mixture in the convex hull?}

The returned $Q_{(m)}$ is in the convex hull. 
The first step returned a single element in $\mathcal{Q}$ with the smallest KL divergence.
Iteratively, the linear combination is still in the convex hull, i.e.
\begin{align*}
    Q_{(2)} & = \alpha_1 \cdot Q_{(1)} + (1 - \alpha_1)\cdot Q_{(1)}' \\
    Q_{(2)} & = \alpha_2 \cdot Q_{(2)} + (1 - \alpha_2)\cdot Q_{(2)}' \\
            & = \alpha_2 \cdot \alpha_1 \cdot Q_{(1)} +
    \alpha_2 \cdot (1 - \alpha_1) \cdot Q_{(1)}' +
    (1 - \alpha_2)\cdot Q_{(2)}'.
\end{align*}

It is clear that $Q_{(2)}$ or $Q_{(m)}$ would still be a convex combination of
elements in $\mathcal{Q}$.


\subsection{Cisszar Algorithm}

Originally proposed by