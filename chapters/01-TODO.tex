\newpage

\section{TODO}
This is just a playground for me! No need to be too nicely formatted.

\subsection{Questions}

\begin{itemize}
	\item What is the regularity condition in universal inference condition?
	\item Why do we want to have data-dependent significant level $\alpha$
	\item What is the difference between \p-process and \E-process and nonnegative martingale?
	\item What is a adapted sequence of random sets, random variables
	\item Wald's sequential likelihood ratio test
	\item What is Radon-Nikodym derivative
	\item Try to explain post-hod and ad-hoc in plain English
	\item What would happen if you just multiply \E-variables together? Shouldn't you consider the
	\item What is the evidence in testing? Is likelihood ratio the best one we have for evidence
\end{itemize}

\begin{itemize}
	\item What is the difference between carrying measure and carrying measure?
	\item Write down the lemma/fact that each element in a regular exponential family is continuous
	      with respect to each other and could be used as carrier. Is it a one-to-one mapping?
	\item Mean-value Parameterization Convexity of mean-value spaces and canonical spaces
	\item Loewner ordering
\end{itemize}

\subsection{Filtrations and sigma-fields}

\begin{define}[Filtration]
	Let $(\Omega, \Ac, P)$ be a probability space. An increasing sequence $\Fc_n$ of sub-$\sigma$ algebras of $\Ac$
	(i.e. $\Fc_0 \subseteq \Fc_1 \subseteq \cdots \Fc_n \subseteq \cdots \subseteq \Ac$) is called a filtration.
\end{define}

The sub-$\sigma$-algebra is just the sigma algebra of the $X_0, X_1, X_n$.

What is the difference between the stopping time and random times that can take possibly infinity.

Filtration sigma-field at time $t$, filtration is a n increasing sequence of sigma-fields.

What is the lim sup here? $A_t$ is an adapted sequences of events in some filtered probability space.
$$
	A_\infty := \limsup_{t\to \infty} A_t := \bigcap_{t\in\Nf}\bigcup_{s\geq t}A_s
$$

\begin{define}[Absolute continuous]
	Let $p, q$ be two probability distributions. $p$ is called \emph{absolutely continuous} with respect to $q$ if the

	More explicitly, this reads ($p \ll q$) that
	\begin{align*}
		q(A) = 0 \quad \Rightarrow \quad p(A) = 0,  \quad \text{for any } A \in \Fc_t \text{ and } t \in \Nf.
	\end{align*}
\end{define}

\subsection{All the inequalities}

Hoeffding, Bernstein, McDiarmid, Talagrand's

Evidence lower bound

line crossing inequality
average treatment effect
mixture adaptive design

\subsection{Papers, talks, textbook, and more topics}

Clarke Barron 1990 1994 about
\begin{itemize}
	\item Stein's Methods
	\item Ito's process
	\item Levy's process
	\item Gaussian process
\end{itemize}

\subsection{Basic Concepts}

\begin{define}[Admissibility]
	In \E-process, a process is inadmissible
	if there exists a process that is strictly better than it at certain time points.
\end{define}

Calibration


Understanding the Fourier transform in terms of transform
Placing a restriction on the Fourier transform == smaller function space
Minimax setup

Integration / Measure

Topology crash course

The topologist Stephen Smale stunned the mathematical community in
1958 [Sma58] when he proved it was possible to turn the sphere inside out
without introducing any creases. Several ways to do this are beautifully
illustrated in video recordings [Max77, LMM94, SFL98].

Tower properties




