\clearpage

\section{Integrals}


What is it other than the area under curve?

Interesting examples where the Riemann's integration fails is that
$f(x) = 1$ if $x$ is rational and $0$ otherwise over the interval $[0,1]$. 
Why this function's integral is problematic in Riemann's definition.

Lebesgue used measure theory to define integral.

\begin{quote}
    Many times, when Riemann fails, Lebesgue works.
\end{quote}

Set being countable and uncountable rely on if we can arrange them in
a `readable' sequence, could be infinite like rational numbers.

Proving that $\Rf$ is uncountable is not so trivial: decimal expansion.

No matter how hard you try to write the numbers in $[0,1)$, 
there is always some values left out.

Continuum hypothesis: How do you check which $\infty$ is bigger? 
Or which set has the higher cardinality?

Definition 1.2.2 ()

\begin{example}[Set of Lebesgue measure zero, 1.2.2]
    A set $S$ is zero in Lebesgue measure 
    when it can be covered with a sequence of open intervals ($I_1, I_2, \dots$). 
    And the sum $\sum_{0}^{\infty} m(I_n)$ can be made arbitrarily small
\end{example}

In other words, for any $\epsilon > 0$, we can always find a sequence of $I_n$ 
that covers $S$.

\begin{thm}
    Any countable infinite set of $S$ has Lebesgue measure zero.
\end{thm}

\begin{remark}
    The set of measure zero would not change a damn thing on $f$
\end{remark}
