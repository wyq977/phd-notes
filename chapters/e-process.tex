\clearpage

\section{Nonnegative martingales and E-process}

\subsection{Stuff}
Being integrable or not, I think, is the main contribution from \cite{wangExtendedVillesInequality2024}.

\begin{question}
	Which class of random processes is bigger? \E-processes or martingales?
\end{question}

If the e-process is defined by the stuff upper bounded NNSM, then martingales is a bigger?
Or should I say for a set of distribution $\Qc$ we can always find some NNSM
upper bounding the $e_t$??

\subsection{Notation}
We limit ourself to discrete time only $n\in\Nf_0=\{0,1,2\dots\}$, usually written as $n\geq 0$.
Fix a probability triplet $(\Omega,\Ac,\Pc)$ and $\Fc_t$, a sigma-field at time $t$, defines
a filtration, $\Fc_0\subseteq\Fc_1\subseteq\Fc_2\subseteq\cdots\Ac$.
We call $\Fc_t$ the canonical filtration and $X=\{X_n\}_{n\geq 0}$ a stochastic
process adapted to $\Fc$ if $X_n$ is $\Fc_n$-measurable.

\begin{define}
	An \E-process for a family $\Pc$ of probability measures, namely $\Pc$-\E-process,
	is a nonnegative process $E$, adapted to some filtrations such that
	\begin{align*}
		\ERWi{\Pt}{E_{\tau}} \leq 1 \qq{for every stopping time} \tau
	\end{align*}
\end{define}

It can also be defined as any nonnegative process that is upper bounded,
for every $P\in\Pc$, by a $P$-martingale starting at one.